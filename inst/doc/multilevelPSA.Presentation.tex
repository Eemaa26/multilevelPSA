%\documentclass[handout]{beamer}
%\documentclass[handout,10pt,slidestop,mathserif]{beamer}
%\usepackage{pgfpages}
%\pgfpagesuselayout{2 on 1}
\documentclass[10pt,slidestop,mathserif]{beamer}
\usetheme{Madrid}
%\usepackage[bars]{beamerthemetree}
\usecolortheme{seahorse}

\usepackage{tabularx}
\usepackage{verbatim}
\usepackage{graphics}
\usepackage{graphicx}
\usepackage{Sweave}
\usepackage{moreverb}
\usepackage{pgf}
\usepackage{tikz}
\usepackage{sparklines}

\newcommand{\putat}[3]{\begin{picture}(0,0)(0,0)\put(#1,#2){#3}\end{picture}}
  
\newenvironment{changemargin}[2]{%
  \begin{list}{}{%
    \setlength{\topsep}{0pt}%
    \setlength{\leftmargin}{#1}%
    \setlength{\rightmargin}{#2}%
    \setlength{\listparindent}{\parindent}%
    \setlength{\itemindent}{\parindent}%
    \setlength{\parsep}{\parskip}%
  }%
  \item[]}{\end{list}}

%% Define a new 'leo' style for the package that will use a smaller font.
\makeatletter
\def\url@leostyle{%
  \@ifundefined{selectfont}{\def\UrlFont{\sf}}{\def\UrlFont{\tiny\ttfamily}}}
\makeatother

\title[MultilevelPSA]{\texttt{multilevelPSA}: An R Package for Estimating and Visualizing Multilevel Propensity Score Models}
\subtitle{}
\author[Bryer]{Jason Bryer}
\institute[University at Albany]{University at Albany}
\date{November 14, 2011}


\begin{document}


\frame{\titlepage}
\frame{\frametitle{Agenda}\tableofcontents[hideallsubsections]}

\section{Overview}

\begin{frame}[containsverbatim,fragile,c]
	\frametitle{Installing multilevelPSA}
	The \texttt{multilevelPSA} package is currently under development and is available on \href{http://github.com}{github}. The \texttt{devtools} package provides a function to install R packages directly from github.
\begin{Schunk}
\begin{Sinput}
> library(devtools)
> install_github('multilevelPSA', 'jbryer')
\end{Sinput}
\end{Schunk}
\pause
Once installed from github, it can be loaded just like any other R package.
\begin{Schunk}
\begin{Sinput}
> library(multilevelPSA)
> ls('package:multilevelPSA')
\end{Sinput}
\begin{Soutput}
 [1] "GeomRugAlt"                   "geom_rug_alt"                
 [3] "getPropensityScores"          "getStrata"                   
 [5] "missingPlot"                  "multilevelCtree"             
 [7] "multilevelLR"                 "multilevelPSA"               
 [9] "plot.multilevel.distribution" "plotcirc.multilevel.psa"     
[11] "plotpsa.multilevel.psa"       "treeHeat"                    
\end{Soutput}
\end{Schunk}
\end{frame}

\begin{frame}[containsverbatim,fragile,c]
	\frametitle{PISA: Programme for International Student Assessment}
	The  Economic Co-operation and Development (OECD) began assessing student achievement in 2003 with the Programme of International Student Assessment (PISA; \url{http://www.pisa.oecd.org/}). In 2009 they evaluated students at the end of secondary school (or equivalent depending on country) in mathematics, reading, and science. Data is freely available on their website but an R data is made available with the \texttt{multilevelPSA} package. There are two data frames, \texttt{student.orig} and \texttt{school.orig} and are loaded using the \texttt{data} command.
\begin{Schunk}
\begin{Sinput}
> data(pisa.student)
> #names(student.orig)
> nrow(student.orig)
\end{Sinput}
\begin{Soutput}
[1] 475460
\end{Soutput}
\begin{Sinput}
> ncol(student.orig)
\end{Sinput}
\begin{Soutput}
[1] 305
\end{Soutput}
\begin{Sinput}
> data(pisa.school)
> #names(school.orig)
> nrow(school.orig)
\end{Sinput}
\begin{Soutput}
[1] 17145
\end{Soutput}
\begin{Sinput}
> ncol(school.orig)
\end{Sinput}
\begin{Soutput}
[1] 247
\end{Soutput}
\end{Schunk}
\end{frame}

\begin{frame}[containsverbatim,fragile,c]
	\frametitle{Covariates}
% latex table generated in R 2.14.0 by xtable 1.6-0 package
% Sat Nov 12 14:24:45 2011
\begin{table}[ht]
\begin{center}
\begin{tabular}{lll}
  \hline
Variable & ShortDesc & Desc \\ 
  \hline
CNT & CNT & Country \\ 
  SCHOOLID & SchoolId & SchoolID \\ 
  StIDStd & StudentId & Student ID \\ 
  ST01Q01 & Grade & Grade \\ 
  ST04Q01 & Sex & Sex \\ 
  ST05Q01 & Attend & Attend \\ 
  ST06Q01 & Age & Age \\ 
  ST07Q01 & Repeat & Repeat \\ 
  ST08Q01 & Mother & At home mother \\ 
  ST08Q02 & Father & At home father \\ 
  ST08Q03 & Brother & At home brothers \\ 
  ST08Q04 & Sister & At home sisters \\ 
  ST08Q05 & GrandPa & At home grandparents \\ 
  ST08Q06 & Other & At home others \\ 
  ST10Q01 & MomEd & Mother highest schooling \\ 
  ST12Q01 & MomJob & Mother current job status \\ 
   \hline
\end{tabular}
\caption{Covariates Used for Propensity Score Estimations}
\end{center}
\end{table}
\end{frame}
\begin{frame}[containsverbatim,fragile,c]
	\frametitle{Covariates (cont.)}
% latex table generated in R 2.14.0 by xtable 1.6-0 package
% Sat Nov 12 14:24:45 2011
\begin{table}[ht]
\begin{center}
\begin{tabular}{lll}
  \hline
Variable & ShortDesc & Desc \\ 
  \hline
ST14Q01 & DadEd & Father highest schooling \\ 
  ST16Q01 & DadJob & Father current job status \\ 
  ST19Q01 & Lang & Language at home \\ 
  ST20Q01 & Desk & Desk \\ 
  ST20Q02 & OwnRoom & Own room \\ 
  ST20Q03 & StudyPl & Study place \\ 
  ST20Q04 & Computer & Computer \\ 
  ST20Q05 & Software & Software \\ 
  ST20Q06 & Internet & Internet \\ 
  ST20Q07 & Lit & Literature \\ 
  ST20Q08 & Poetry & Poetry \\ 
  ST20Q09 & Art & Art \\ 
  ST20Q10 & TxtBooks & Textbooks \\ 
  ST20Q12 & Dict & Dictionary \\ 
  ST20Q13 & DishW & Dishwasher \\ 
  ST20Q14 & DVD & DVD \\ 
   \hline
\end{tabular}
\caption{Covariates Used for Propensity Score Estimations}
\end{center}
\end{table}
\end{frame}
\begin{frame}[containsverbatim,fragile,c]
	\frametitle{Covariates (cont.)}
% latex table generated in R 2.14.0 by xtable 1.6-0 package
% Sat Nov 12 14:24:45 2011
\begin{table}[ht]
\begin{center}
\begin{tabular}{lll}
  \hline
Variable & ShortDesc & Desc \\ 
  \hline
ST21Q01 & CellPh & How many cellphones \\ 
  ST21Q02 & TVs & How many TVs \\ 
  ST21Q03 & nComp & How many computers \\ 
  ST21Q04 & nCars & How many cars \\ 
  ST21Q05 & nBaths & How many rooms bath or shower \\ 
  ST22Q01 & nBooks & How many books \\ 
  ST23Q01 & Reading & Reading enjoyment time \\ 
  ST31Q01 & EnrichLang & Enrich in test language \\ 
  ST31Q02 & EnrichMath & Enrich in mathematics \\ 
  ST31Q03 & EnrichScie & Enrich in science \\ 
  ST31Q05 & RemedialLang & Remedial in test language \\ 
  ST31Q06 & RemedialMath & Remedial in mathematics \\ 
  ST31Q07 & RemedialScie & Remedial in science \\ 
  ST32Q01 & LangLessons & Out of school lessons in test language \\ 
  ST32Q02 & MathLessons & Out of school lessons maths \\ 
  ST32Q03 & ScieLessons & Out of school lessons in science \\ 
   \hline
\end{tabular}
\caption{Covariates Used for Propensity Score Estimations}
\end{center}
\end{table}
\end{frame}

\begin{frame}[containsverbatim,fragile,c]
	\frametitle{Setup School Data}
\begin{Schunk}
\begin{Sinput}
> school = school.orig[,c('COUNTRY', "CNT", "SCHOOLID",
+ 	"SC02Q01", #Public (1) or private (2)
+ 	"STRATIO" #Student-teacher ration    
+ )]
> names(school) = c('COUNTRY', 'CNT', 'SCHOOLID', 'PUBPRIV', 'STRATIO')
> school$SCHOOLID = as.integer(school$SCHOOLID)
\end{Sinput}
\end{Schunk}
\end{frame}

\begin{frame}[containsverbatim,fragile,c]
	\frametitle{Number of Private and Public Schools by Country}
<<results=tex>>
t = table(school$COUNTRY, school$PUBPRIV, useNA='ifany')
x = xtable(t[1:22,], caption='Number of Private and Public Schools by Country', label='ppxtab')
print(x, include.rownames=FALSE, include.colnames=TRUE)

\end{frame}

\begin{frame}[containsverbatim,fragile,c]
	\frametitle{}
\end{frame}

\begin{frame}[containsverbatim,fragile,c]
	\frametitle{}
\end{frame}

\begin{frame}[containsverbatim,fragile,c]
	\frametitle{}
\end{frame}


\begin{frame}[containsverbatim,fragile,c]
	\frametitle{}
\end{frame}

{ % Figure that fills the frame
    \setbeamertemplate{navigation symbols}{}
    \begin{frame}[plain]
        \begin{tikzpicture}[remember picture,overlay]
            \node[at=(current page.center)] {
               	
            };
        \end{tikzpicture}
     \end{frame}
}



\section{Conclusions \& Questions}
\begin{frame}[c]
	\LARGE{Thank You}\\
	\normalsize
	Jason Bryer (jason@bryer.org)\\
	\url{https://github.com/jbryer/multilevelPSA}
\end{frame}

\end{document}
